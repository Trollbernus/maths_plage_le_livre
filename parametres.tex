%---------------------------
% parametres classicalthesis
%---------------------------

\usepackage[T1]{fontenc}
\usepackage[utf8]{inputenc} %Permet d'utiliser des caractères spéciaux : éèàçù etc. plus sûr que \usepackage[latin9]{inputenc}
\usepackage{classicthesis/classicthesis}
\PassOptionsToPackage{
  drafting=false,    % print version information on the bottom of the pages
  tocaligned=false, % the left column of the toc will be aligned (no indentation)
  dottedtoc=false,  % page numbers in ToC flushed right
  eulerchapternumbers=true, % use AMS Euler for chapter font (otherwise Palatino)
  linedheaders=false,       % chaper headers will have line above and beneath
  floatperchapter=true,     % numbering per chapter for all floats (i.e., Figure 1.1)
  eulermath=false,  % use awesome Euler fonts for mathematical formulae (only with pdfLaTeX)
  beramono=true,    % toggle a nice monospaced font (w/ bold)
  palatino=true,    % deactivate standard font for loading another one, see the last section at the end of this file for suggestions
  style=classicthesis % classicthesis, arsclassica
}{classicthesis}
\providecommand{\mLyX}{L\kern-.1667em\lower.25em\hbox{Y}\kern-.125emX\@}
%\usepackage{euler}
% ****************************************************************************************************
\usepackage{tabularx} % better tables
\setlength{\extrarowheight}{3pt} % increase table row height
\newcommand{\tableheadline}[1]{\multicolumn{1}{l}{\spacedlowsmallcaps{#1}}}
\newcommand{\myfloatalign}{\centering} % to be used with each float for alignment
\usepackage{subfig}

\usepackage{listings}
%\lstset{emph={trueIndex,root},emphstyle=\color{BlueViolet}}%\underbar} % for special keywords
\lstset{language=[LaTeX]Tex,%C++,
  morekeywords={PassOptionsToPackage,selectlanguage},
  keywordstyle=\color{RoyalBlue},%\bfseries,
  basicstyle=\small\ttfamily,
  %identifierstyle=\color{NavyBlue},
  commentstyle=\color{Green}\ttfamily,
  stringstyle=\rmfamily,
  numbers=none,%left,%
  numberstyle=\scriptsize,%\tiny
  stepnumber=5,
  numbersep=8pt,
  showstringspaces=false,
  breaklines=true,
  %frameround=ftff,
  %frame=single,
  belowcaptionskip=.75\baselineskip
  %frame=L
}

\listfiles

%---------------------------
% paquets persos
%---------------------------

\usepackage[french]{babel}


\usepackage{pdfpages} % pour inclure des pdf

\usepackage{graphicx} %??
\usepackage{amssymb} %??

\usepackage{caption} %??
\usepackage{amsmath, amssymb, amsthm} % maths
\usepackage{natbib}
\setcitestyle{square}

\usepackage{colortbl} 

%\usepackage[french]{minitoc} % on sait jamais...
\usepackage{tocloft}  
\usepackage{titletoc}  
\usepackage{appendix} 
\usepackage{eurosym}
\usepackage{stmaryrd} %crochets d'entiers
\usepackage{empheq} % équations encadrées
\usepackage{mathrsfs} % zoulies majuscules
\usepackage{pgf,tikz} % dessiner avec latex
\usetikzlibrary{arrows}
\usepackage{yhmath} % arc de cercle sur lettres : \wideparen{AB}
\usepackage{bm} % gras en maths
\usepackage{hhline} % allows double hline
\usepackage{lscape} % pour mettre des pages en paysage
\usepackage{cancel} % barrer ds equation
\usepackage{multirow} % tableau multicolonne/ligne
\usepackage{multicol}
%\usepackage{mathabx} % symboles astro

%\usepackage[footnotesize]{subcaption} %for subfigures
\usepackage{adjustbox}

% utile si on veut rapidement mettre un truc en couleur. Modifier black par la valeur souhaitée.
\newcommand{\colp}[1]{\textcolor{black}{#1}}


\usepackage[shortlabels]{enumitem}

\usepackage{xcolor}
\usepackage{color} 
\definecolor{ffqqqq}{rgb}{1,0,0}
\definecolor{aaaaaa}{rgb}{1,1,1}
\definecolor{qqqqff}{rgb}{0,0,1}
\definecolor{xdxdff}{rgb}{0.49,0.49,1}
\definecolor{FF6B00}{rgb}{0.5,0,0}
\definecolor{dblue}{rgb}{0,0.3,0.5}
\definecolor{dgreen}{rgb}{0,0.5,0.4}

\usepackage{etoolbox}

\usepackage{hyperref}
\hypersetup{%
  %draft, % hyperref's draft mode, for printing see below
  colorlinks=true, linktocpage=true, pdfstartpage=3, pdfstartview=FitV,%
  % uncomment the following line if you want to have black links (e.g., for printing)
  %colorlinks=false, linktocpage=false, pdfstartpage=3, pdfstartview=FitV, pdfborder={0 0 0},%
  breaklinks=true, pageanchor=true,%
  pdfpagemode=UseNone, %
  % pdfpagemode=UseOutlines,%
  plainpages=false, bookmarksnumbered, bookmarksopen=true, bookmarksopenlevel=1,%
  hypertexnames=true, pdfhighlight=/O,%nesting=true,%frenchlinks,%
  urlcolor=CTurl, linkcolor=CTlink, citecolor=CTcitation, %pagecolor=RoyalBlue,%
  %urlcolor=Black, linkcolor=Black, citecolor=Black, %pagecolor=Black,%
  pdftitle={maths-plage},%
  pdfauthor={Loïc et Léo},%
  pdfsubject={},%
  pdfkeywords={},%
  pdfcreator={pdfLaTeX},%
  pdfproducer={LaTeX with hyperref and classicthesis}%
}

\makeatletter

%Pour colorer les crochets des citations
\pretocmd{\NAT@open}{\begingroup\color{\@citecolor}}{}{}
\apptocmd{\NAT@close}{\endgroup}{}{}


%Pour colorer les parenthèses des équations citées.
\let\oldtheequation\theequation
\renewcommand\tagform@[1]{\maketag@@@{\ignorespaces#1\unskip\@@italiccorr}}
\renewcommand\theequation{(\oldtheequation)}

% pour mettre des lignes de commandes dans le texte. 
% Attention, il faut ajouter l'option --shell-escape après pdflatex. 
\usepackage{minted}
\usemintedstyle{vs}

% pour indiquer des changements.
\usepackage{changes}
\definechangesauthor[color=BrickRed]{LC}
\definechangesauthor[color=blue]{LB}
\newcommand{\addlb}[1]{\added[id=LB]{ #1}}
\newcommand{\replb}[2][]{\replaced[id=LB]{#1}{#2}}
\newcommand{\deletb}[1]{\deleted[id=LB]{#1}}
\newcommand{\addlc}[1]{\added[id=LC]{ #1}}
\newcommand{\replc}[2][]{\replaced[id=LC]{#1}{#2}}
\newcommand{\deletlc}[1]{\deleted[id=LC]{#1}}


%--------------------------------------
% raccourcis pour des symboles utiles
%--------------------------------------

% je laisse ça là
\def\eps{\varepsilon}
\def\percent{\%}
\def\etal{{et al.}}



%%%%%%%%%%%Commandes pour la bibliographie.
\def\aj{\textit{The Astronomical Journal}}
\def\nat{\textit{Nature}}
\def\icarus{\textit{Icarus}}
\def\aap{\textit{Astronomy and Astrophysics}}
\def\grl{\textit{Geophysical Research Letters}}
\def\jgr{\textit{Journal of Geophysical Research}}
\def\prd{\textit{Physical Review D}}
\def\prl{\textit{Physical Review Letter}}
\def\mnras{\textit{Monthly Notices of the Royal Astronomical Society}}
\def\apjl{\textit{Astrophysical Journal Letters}}
\def\apj{\textit{Astrophysical Journal}}
\def\physrep{\textit{Physics Reports}}





\makeatother



\newcommand{\HRule}{\rule{\linewidth}{0.5mm}}



%%%%%%%%%%%%%%%%%%%%%%%%%%%%%%%%%%%%%%%%%%ù
%
%typographie matheuse
%
%%%%%%%%%%%%%%%%%%%%%%%%%%%%%%%%%%%%%%%%%%

%\newcommand{\d}{\mathop{}\mathrm{d}}
\def\lg{\lambda_g}
\def\d{\mathrm{d}}
\def\e{\mathrm{e}}
\def\i{\mathrm{i}}
\def\R{\mathbb{R}}
\def\Q{\mathbb{Q}}
\def\D{\mathbb{D}}
\def\Z{\mathbb{Z}}
\def\P{\mathbb{P}}
\def\ch{\mathrm{ch}}
\def\sh{\mathrm{sh}}
\def\M{\mathcal{M}}
\def\N{\mathbb{N}}
\def\R{\mathbb{R}}
\def\Z{\mathbb{Z}}
\def\B{\mathscr{B}}
\def\cs{\xrightarrow[]{c.s.}}
\def\cu{\xrightarrow[]{c.u.}}
\def\cL1{\xrightarrow[]{L^1}}
\def\cLp{\xrightarrow[]{L^p}}
\def\({\left(}
\def\){\right)}
\def\[{\left[}
\def\]{\right]}
\def\EDO{équation différentielle ordinaire}
\def\EDOU{équation différentielle ordinaire d'ordre 1}
\def\ch{\mathrm{ch\,}}
\def\sh{\mathrm{sh\,}}
\def\ach{\mathrm{ach\,}}
\def\ash{\mathrm{ash\,}}
\def\ddet{\mathrm{det\,}}
\def\ninN{n\in\mathbb{N}}
\def\asin{\mathrm{asin\,}}
\def\acos{\mathrm{acos\,}}
\def\atan{\mathrm{atan\,}}
\def\psd{\frac{\pi}{2}}
\def\convn{\xrightarrow[n\to\infty]{}}
\def\C{\mathbb{C}}
\def\S{\mathscr{S}}
\newcommand{\ve}[1]{\overrightarrow{\bm #1}}
\newcommand{\fl}[1]{\underline{\bm #1}}
\newcommand{\de}[1]{\underline{\bm{\d}\bm #1}}
\newcommand{\ovr}[1]{\overrightarrow{#1}}
\newcommand{\vei}[2][]{\overrightarrow{\bm{#1}_{#2}}}
\newcommand{\guil}[1]{\og{} #1\fg}

%%%%%%%%%%%%%%%%%%%%%%%%%%%%%%%%%%%%%%%%%%%%%
%--------------------------------------------
%   Commandes de Loïc (légèrement modifiées)
%--------------------------------------------
%%%%%%%%%%%%%%%%%%%%%%%%%%%%%%%%%%%%%%%%%%%%%


% --------------------------------------
% Commande chiffre Romain
% --------------------------------------
\newcommand{\Rom}[1]{\uppercase\expandafter{\romannumeral #1\relax}}

% --------------------------------------
% Initial de texte
% --------------------------------------
\usepackage{lettrine}
\newcommand{\initial}[2]{\lettrine[lines=2, lhang=0., loversize=0.1,findent=0.2em, nindent=0pt]{\color{BrickRed}#1}{#2}}

% --------------------------------------
% Couleurs perso
% --------------------------------------
\definecolor{souris}{gray}{0.95}
\definecolor{LightGrey}{gray}{0.92}



% --------------------------------------
% Math : ensemble notation famille d'élèments cardinal
% --------------------------------------
\newcommand{\ensemble}[2]{\left\{#1\,\left|\,#2\right.\right\}}
\newcommand{\fami}[3]{ \left(#1_{#2}\right)_{#2 = #3}}
\newcommand{\famiin}[3]{ \left(#1_{#2}\right)_{#2 \in #3}}
\newcommand{\famiup}[3]{ \left(#1^{#2}\right)_{#2 = #3}}
\newcommand{\famiupin}[3]{ \left(#1^{#2}\right)_{#2 \in #3}}
\newcommand{\card}{\text{card}}

% --------------------------------------
% Angles
% --------------------------------------
\newcommand\angdroit{%
    \tikz[line cap=round,x=1.6ex,y=1.6ex,line width=0.4pt]
    {\draw (0,1) |- (1,0) (0,0) rectangle (.55,.55);}
}
\newcommand\angqcq{%
    \tikz[line cap=round,x=1.6ex,y=1.6ex,line width=0.3pt]
    {\draw (1,0) -- (0,0); \draw[rotate = 75] (1,0) -- (0,0); \draw (0.55,0) arc(0:75:0.55);}%
    }
\newcommand\angbar{%
    \tikz[line cap=round,x=1.6ex,y=1.6ex,line width=0.3pt]
    {\draw (1,0) -- (0,0); \draw[rotate = 75] (1,0) -- (0,0); \draw (0.55,0) arc(0:75:0.55); \draw[rotate = 37.5] (0.30,0) -- (0.80,0) }%
    }
\newcommand\angdoublebar{%
    \tikz[line cap=round,x=1.6ex,y=1.6ex,line width=0.3pt]
    {\draw (1,0) -- (0,0); \draw[rotate = 75] (1,0) -- (0,0); \draw (0.55,0) arc(0:75:0.55); \draw[rotate = 37.5] (0.35,0.08) -- (0.80,0.08) (0.35,-0.08) -- (0.80,-0.08)}%
    }
\newcommand\angcircle{%
    \tikz[line cap=round,x=1.6ex,y=1.6ex,line width=0.3pt]
    {\draw (1,0) -- (0,0); \draw[rotate = 75] (1,0) -- (0,0); \draw (0.55,0) arc(0:75:0.55); \draw[rotate = 37.5] (0.55,0.0) circle (0.20)}%
    }

\newcommand\angtroispoint[3]{%
    \vcenter{\hbox{\tikz[line cap=round,x=1.6ex,y=1.6ex,line width=0.3pt]
    {\draw (0,0) node[left]{$#2$};
    \draw[rotate = -30] (0.5,0) -- (1.8,0); 
    \draw[rotate = -30] (2.5,0) node[right]{$#1$};
    \draw[rotate = 30] (0.5,0) -- (1.8,0);
    \draw[rotate = 30] (2.5,0) node[right]{$#3$};
    \draw[domain=-30:30] plot ({cos(\x)}, {sin(\x)});}%
    }}
}

\newcommand\anglesaillant{%
    \vcenter{\hbox{\tikz[line cap=round,x=1.6ex,y=1.6ex,line width=0.3pt]
    {\draw[rotate = -30] (0.0,0) -- (1.5,0); 
    \draw[rotate = 30] (0.0,0) -- (1.5,0);
    \draw [domain=-30:30] plot ({1.1*cos(\x)}, {1.1*sin(\x)});}%
    }}
}



\newcommand\anglerentrant{%
    \vcenter{\hbox{\tikz[line cap=round,x=1.6ex,y=1.6ex,line width=0.3pt]
    {\draw[rotate = -30] (0.0,0) -- (1.5,0); 
    \draw[rotate = 30] (0.0,0) -- (1.5,0);
    \draw [domain=30:330] plot ({0.8*cos(\x)}, {0.8*sin(\x)});}%
    }}
}

\newcommand\angrentrtroispoint[3]{%
    \vcenter{\hbox{\tikz[line cap=round,x=1.6ex,y=1.6ex,line width=0.3pt]
    {\draw (-0.35,0) node{#2};
    \draw[rotate = -30] (0.5,0) -- (1.8,0); 
    \draw[rotate = -30] (2.5,0) node{#1};
    \draw[rotate = 30] (0.5,0) -- (1.8,0);
    \draw[rotate = 30] (2.5,0) node{#3};
    \draw [domain=30:330] plot ({1.25*cos(\x)}, {1.25*sin(\x)});}%
    }}
}

% --------------------------------------
% Math : raccourcit pour phrases type
% --------------------------------------
\newcommand{\cad}{c'est-à-dire }
\newcommand{\cst}{{\rm Cst} }
\newcommand{\ssi}{si et seulement si }
%\newcommand{\guil}[1]{\og{} #1\fg} % entourer de guillemets % deja definie plus haut

\setcounter{secnumdepth}{3}

% remarque : j'ai récrit ce truc 
\newcommand\FigureMarge[3]
{\marginpar{#1 
    
    \vspace{0.1cm}

    \includegraphics[width=#2\marginparwidth]{#3}
}}


% inutile mais je le garde au cas où.
% \title{\color{BrickRed}{\huge Maths plage} \\\phantom{t} \\École de mathématiques pour non-mathématiciens}


% \author{Loïc Chantry - Léo Bernus}
% \date{Été 2023}



%-------------------------------------------
% Gestion des theoremes
%-------------------------------------------

\usepackage{amsthm}
  \newtheorem{post}{Postulat}[section]
  \newtheorem{thm}{Théorème}[section]
  \newtheorem{lem}{Lemme}[section]
  \newtheorem{defi}{Définition}[section]
  \newtheorem{cor}{Corollaire}[section]
  \newtheorem{prop}{Proposition}[section]
  \newtheorem{conj}{Conjecture}[section]
  \newtheorem*{post*}{Postulat}
  \newtheorem*{axiom*}{Axiome}
  \newtheorem*{theorem*}{Théorème}
  \newtheorem*{lemma*}{Lemme}
  \newtheorem*{definition*}{Définition}
  \newtheorem*{cor*}{Corollaire}
  \newtheorem*{prop*}{Propriété}
  \newtheorem*{conj*}{Conjecture}

% coloration et numérotation spéciale pour les axiomes
\usepackage[many]{tcolorbox}
\newtheorem{axi}{Axiome}[subsection]
\tcolorboxenvironment{axi}{
  colback=grey,
  boxrule=0pt,
  boxsep=0pt,
  left=4pt,right=4pt,top=4pt,bottom=4pt,
  oversize=4pt,
  sharp corners,
  before skip=\topsep,
  after skip=\topsep,
}
\newcounter{serieaxiom}
\setcounter{serieaxiom}{1}
\newcounter{axinum}
\setcounter{axinum}{1}
\renewcommand{\theaxi}{\Alph{serieaxiom}.\arabic{axi}}

%\theorembodyfont{\upshape}
