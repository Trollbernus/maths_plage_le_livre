
\chapter*{Introduction}
\addcontentsline{toc}{chapter}{\protect\numberline{}Introduction}
\markboth{Introduction}{}

\initial{D}{ans} la \guil{République des valeurs}, l'\guil{école de la confiance} produit les travailleurs de demain. La conséquence logique est que l'enseignement des mathématiques dans cette institution, dans sa forme et dans son contenu, est essentiellement déterminé par les besoins de la classe dirigeante : sélection sociale et utilitarisme vis à vis des activités concrètes des forces productives requises. Dans un contexte de désindustrialisation massive, cela implique également une destruction de la pensée mathématique à l'école : la logique déductive fait place au parcœurisme des formules magiques pour fabriquer en masse des esclaves de servitude volontaire.
Heureusement, ici nous ne sommes pas dans cette école.

Notre but pédagogique est de donner des outils et une méthode de réflexion et de logique, pour que nos élèves augmentent leurs capacités de compréhension du monde réel qui les entoure, afin de pouvoir agir sur lui. 
Lors de la première session \footnote{Il s'agissait de Maths-champignons (automne 2022) : \url{https://perso.crans.org/lbernus/leobernus/index.php?section=enseignement\#mathchampis}}, 
nous nous sommes concentrés sur les fondements des mathématiques : la théorie des nombres, les axiomes de la géométrie plane, et la logique mathématique elle-même. 
Partant de ces bases, nous allons pouvoir utiliser ces nombres et ces formes géométriques pour résoudre quelques problèmes de mathématiques. Ainsi, les mathématiques apparaîtront comme un formidable outil théorique pour comprendre le mouvement de la matière. Vous l'aurez compris, cette deuxième année ouvre légèrement le champ d'étude à la science physique. Cette science s'occupe de comprendre comment des grandeurs mesurables et variables sont reliées entre elles, en établissant une relation déterministe et univoque. Pour ce faire, les savants utilisent abondamment les notions mathématiques d'équations et de fonctions. C'est pourquoi ces deux objets seront le cœur de l'étude de cette deuxième session.

Cette école d'été s'inscrit dans l'éducation populaire prodiguée par la FFCC (Fédération Française des Concours de Circonstances, association loi 1901), elle est donc gratuite et ouverte à tous. Il est possible de suivre cette deuxième session sans avoir suivi la première ; des rappels seront faits et il n'y a pas de prérequis. Si vous vous sentez "nul en maths" voire "catastrophique", cette école est spécialement faite pour vous.

Ce document constitue les notes de cours qui nous ont servi de base de travail pour préparer les cours. Rien ne peut remplacer les séances orales des cours qui entraînent les individus à pratiquer les mathématiques, mais rien ne peut non plus remplacer la précision et le détail d'un écrit mathématique. Ce texte, en plus de constituer un aide-mémoire, contient de nombreux détails que nous n'avons pas eu le temps de traiter pendant les séances, que les élèves ambitieux apprécieront.

Le texte est divisé en deux parties. La partie \ref{part_1}, en guise de prélude, permet aux élèves de se réhabituer à manipuler des expressions algébriques élémentaires. On y aborde ce qui est vulgairement appelé \guil{produit en croix} (chapitre \ref{chap_croix} au collège, on réapprend à résoudre des équations du premier degré \ref{chap_eqp}. Même si l'approche est semi-intuitive (la notion de nombres réels n'est pas introduite de manière rigoureuse), tous les résultats sont démontrés à partir d'axiomes d'algèbre élémentaire, par exemple, que si à deux quantités égales on en retire une même troisième, les résultats des deux opérations seront égaux. C'était l'approche d'Euclide et de Descartes. Dans la résolution des équations du premier degré, nous introduisons la notion de représentation graphique de fonctions afin de pouvoir interpréter géométriquement la résolution d'équations du premier degré. Nous y apprenons donc à repérer un point dans le plan à partir de nombres, et nous étudions comment la position de tels points varient lorsque ces nombres sont reliés par une égalité, et que l'un des deux varie. Les exemples les plus fréquemments utilisés pour illustrer cette partie sont tirés de problèmes de cinématique (étude mathématique du mouvement des corps matériels dans l'espace). 

La partie \ref{part_2} étudie la notion d'angle de manière axiomatique. Avant d'entrer dans le dur des axiomes, dans le chapitre \ref{chap_pinombre} nous introduisons le nombre $\pi$ de manière semi-intuitive. Nous démontrons que $\pi$ est bien un nombre, et nous exposons la méthode d'Archimède pour le calculer\,: il s'agit de subdiviser le cercle en des polygones réguliers qui contiennent de plus en plus de côtés.
Ensuite, nous entrons dans le dur des axiomes, en suivant l'approche de Hilbert [insérer réf]. Après avoir exposé quelques notions premières, nous définissons les notions de secteur d'angle, de mesure de secteur d'angle, et d'angle. Cela permet au lecteur de sortir du brouillard d'indistinction dans lequel l'Éducation Nationale l'a plongée et dans lequel ces trois notions sont confondues dans le mot \guil{angle} qui évoque tout pour le professeur, et rien pour l'élève. Après cela, nous introduisons la notion de rotation orientée, pour arriver à la notion d'angle orienté\,: pour aller vite, il s'agit de savoir dans quel sens on tourne en plus de connaître la valeur de l'angle.
Une fois que ces notions élémentaires ont bien été définies, nous abordons les fonctions trigonométriques dans le chapitre \ref{chap_fcttrigo}. Ces fonctions établissent un lien entre le mouvement de rotation sur le cercle et les coordonnées d'un point mobile sur ce cercle, son angle orienté étant connu. Diverses relations algébriques utiles sont exposées, et des applications à la physique sont abordées. Notamment, nous montrons comment déterminer quantitativement la position d'un point sur une roue qui roule sans glisser, et nous montrons comment calculer la distance entre la Terre et une étoile donnée par parallaxe. 

Une annexe contient quelques suppléments non indispensables mais utiles\,: L'annexe \ref{app_nbs} contient des rappels sur les ensembles de nombres (nombres entiers naturels $\N$, entiers relatifs $\Z$, décimaux $\D$, rationnels $\Q$, réels $\R$, et une petite introduction à l'ensemble des nombres complexes $\C$). L'annexe \ref{app_conv} démontre la convergence de l'algorithme du calcul du nombre $\pi$ dans le chapitre \ref{chap_pinombre}. L'annexe \ref{app_flocon} évoque rapidement un exemple de courbe qui a une longueur infinie mais qui renferme une aire finie. 

Ce texte contient probablement encore des erreurs et des incomplétudes, dans ce cas, les auteurs invitent les lecteurs à les contacter\,:
\begin{itemize}[$\bullet$]
    \item Léo Bernus\,: leo.bernus@gmail.com,
    \item Loïc Chantry\,: loic.chantry@obspm.fr.
\end{itemize}

