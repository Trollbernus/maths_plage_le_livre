\chapter{Introduction générale}

\initial{M}{aintenant}\addlb{ que nous nous sommes habitués à manipuler des expressions algébriques, nous allons passer à une branche des mathématiques qui se situe à l'interface entre la géométrie et l'analyse\,: la trigonométrie circulaire. Cette discipline s'occupe d'étudier les mouvements de rotation de manière calculatoire en utilisant une approche cartésienne. Il s'agit d'étudier l'évolution des coordonnées cartésiennes d'un point qui évolue sur un cercle.}

\addlb{Par traditionalisme obsolète, et aussi par une sorte d'\guil{esprit praticiste} qui méprise la théorie, et donc tout ce qui est sérieux en mathématiques, dans \guil{l'école de la confiance de la République des valeurs}, l'unité de mesure des angles est le degré. La confusion empiriste règne, puisqu'on enseigne la notion d'angle d'abord par la mesure d'angle avec le \guil{rapporteur}, comme si c'était un objet magique, et en faisant abstraction de tout le développement théorique qui permet sa construction. En outre, depuis quelques décennies, les programmes de mathématiques de l'Éducation Nationale entretiennent sciemment la confusion entre les secteurs d'angles (lieux géométriques), la mesure des secteurs d'angles (nombres réels), et les angles, intermédiaires entre les deux\,: un angle est l'ensemble des secteurs d'angles qui ont une même mesure donnée. Sous prétexte de ne pas vouloir être trop exigeant avec les élèves, on entretient un brouillard de confusion entre des objets de nature différente. En outre, considérer les jeunes esprits comme incapables de comprendre la mathématique, n'est-ce pas là le mépris le plus effroyable envers ces esprits\,? C'est ce qu'il convient d'appeler \guil{bienveillance}, aujourd'hui, au sein de l'Éducation Nationale\ldots} 

\addlb{Cela va sans dire --- mais va encore mieux en le disant --- notre approche sera radicalement opposée. Nous n'allons pas utiliser l'unité du degré pour les angles, mais son unité naturelle, le radian. Pour le dire vite, l'unité du radian se définit de telle sorte qu'un tour complet correspond à $2\pi$ radian. Le radian est la longueur de l'arc parcouru sur un cercle de rayon unité lorsqu'un point le parcourt selon un mouvement de rotation. Mais pour ce faire, en préambule, nous allons expliquer ce que représente géométriquement le nombre $\pi$, dans une approche semi-intuitive --- l'essentiel des résultats sera démontré, mais nous admettrons les théorèmes classiques qui permettent de faire les démonstrations. La plupart de ces théorèmes ont été abordés pendant \guil{maths-champignons}. Après cela, nous allons axiomatiser les notions de secteurs d'angles, de mesures d'angles, d'angles, et d'angles orientés, afin de pouvoir enfin aborder les fonctions trigonométriques qui font le lien entre la position d'un point sur un cercle et ses coordonnées cartésiennes. L'axiomatisation suit la méthode de Hilbert [réf] qui a voulu récrire une géométrie euclidienne axiomatisée en évitant les quelques défauts que présentent les \emph{Éléments géométriques} d'Euclide.}

\addlb{On (r)appelle le vocabulaire élémentaire de la théorie des ensembles\,:}
\begin{itemize}[label=\textbullet]
    \item \addlb{$x\in E$ signifie \guil{$x$ est un élément de l'ensemble $E$.}}
    \item \addlb{$\forall$ signifie \guil{pour tout}. Ce symbole est le quantificateur universel, il peut être utilisé pour définir une variable avant de l'utiliser\,: $\forall x\in E, P(x)$ signifie \guil{pour tout $x$ appartenant à l'ensemble $E$, $x$ vérifie la propriété $P$.}}
    \item \addlb{$\exists$ est le quantificateur d'existence. Par exemple, $\forall y\in \R,\exists x\in\R, y=2x$ signifie \guil{pour tout $y$ nombre réel, il existe un nombre réel $x$ tel que $y=2x$.}}
    \item \addlb{Si $A$ et $B$ sont des ensembles, $A-B$ est l'ensemble des éléments de $A$ qui ne sont pas dans $B$, autrement dit, tous les éléments de l'ensemble $A$, sauf ceux qui sont dans $B$.}
    \item \addlc{On notera $:=$ pour signifier égal par définition.}
\end{itemize}
%\remarquemargehistoireimage{On sait peu de chose de la vie d'Euclide sinon qu'il aurai vécu autour du \Romanbar{3}\iemes siècle avant notre ére. Son ouvrage : \href{http://promenadesmaths.free.fr/telecharger/euclide_elements_1804.pdf}{Éléments} contient une tentative quasiment aboutit de construction d'une géométrie axiomatique. Il contient de nombreux théorème et leurs démonstrations. Il connue de nombreuses éditions et les copies successives contiennent des ajouts ou modifications de ceux qui les produire de sorte que le texte qui nous est parvenue n'est pas tout a fait le texte produit par Euclide. On peut néanmoins supposer que ces modifications respectent l'esprit d'Euclide. Son contenue reste au centre de l'enseignement de la géométrie du secondaire. }{image/EuclideGravure.jpg}{-1.5cm}

%\initial{E}{n} 1899, les << \'Eléments >> d'Euclide ne satisfont plus les mathématiciens. Non pour ses éventuelles limitations, mais parce que justement le système d'axiome proposé par Euclide est lacunaire. Plusieurs fois, dans la lecture des \emph{éléments}, on remarque que certains résultats, implicitement considérés comme vrai, sont utilisés sans avoir fait l'objet de démonstrations. Et pour cause, avec les axiomes présents, ils sont indémontrable. Le but de la relecture par Hilbert est de proposer un système d'Axiome suffisant, respectant l'esprit de la construction Euclidienne et permettant d'aboutir à la même géométrie. Il publiera les resultats de ses travaux dans un livre : <<\href{https://www.pedagogie.ac-aix-marseille.fr/upload/docs/application/pdf/2019-11/principes_fondamentaux_geometrie_-_david_hilbert.pdf}{Les principes fondamentaux de la géométrie}>> qui sera traduit et publié en français en 1900.

%En parallèle, la fin de la construction rigoureuse et axiomatique des ensembles de nombres (Péano pour les entiers et les rationnels, Dedekind pour les réels), l'introduction des notions d'espaces vectoriels, l'étude de ses espaces munie d'un produit scalaire (Espace Hilbertien) permettait de comprendre que les espaces dits pré-Hilbertiens réels vérifiait tout les axiomes d' Euclide ou d'Hilbert. Ces espaces étais donc ceux dont la géométrie est Euclidienne. Cette construction vérifie une axiomatique bien différentes, dites affines, enseigné aprés bac.

%Ces deux axiomatiques de la même géométrie (dite Euclidienne) correspondent toute deux à deux conceptions de la géométrie dont l'importance est historiquement déterminé. L'axiomatique à la Euclide, à la Hilbert où même à la Lobatchevskii (géométrie hyperbolique) procède d'une conception, d'un usage qualifié de <<synthétique>>. Cette conception de l'espace se conçoit par elle même, n'utilisant dans ses axiomes et énoncés que des points ou des partie de l'espace, elle concerne très souvent des propriétés globales de ces parties. Le repère Cartésien, et l'introduction des coordonnées permettra progressivement le développement de la conception <<affine>>\footnote{la transition étant assurée par une conception dites <<analytique>>} de la géométrie, transformant les problèmes de géométrie en équation et les points en vecteur. L'algèbre linéaire et les connaissance sur les nombres fournissant les ressources de résolution. On pouvait dés lors re-axiomatiser la géométrie en attachant à l'espace une structure d'espace vectoriel \footnote{et une forme bilinéaire définie positive afin d'y induire une notion de distance et d'orthogonalité} associant à chaque couple de point un vecteur et à un point et un vecteur associant un autre point : l'espace affine. Cette conception associé au calcul différentiel étais alors apte à la généralisation et la formalisation d'une vaste gamme de géométrie non Euclidienne : les variétés différentielles (sans distance) et les variété différentielle de Riemann (avec une distance). Les propriétés y étant souvent exprimés sous forme d'équations différentielles (\cad local), l'intégration permettait alors d'y retrouver des propriétés globales à l'allure synthétique \footnote{voir par exemple le Théorème de Gauss-Bonnet dont l'application sur les surfaces relie somme des angles d'un triangle de la surface et aires de celui-ci}\footnote{voir également la théorie synthétique de la courbure de Ricci dont Cédric Villani à fait une présentation sur youtube}. A mesure que l'apparition de propriété synthétique obtenue à partir des structures <<affine>> des variétés de Riemann s'accumulait l'idée apparaissait de savoir si toute géométrie non-Euclidienne possédait une axiomatisation synthétique (comme celles d'Euclide et de Lobatchevskii). C'est à ce moment qu'une axiomatiqation sérieuse et sans faille des espaces Euclidien eux même devenait de plus en plus urgente. Hilbert s'acquitta de cette tache. D'un coté la conception synthétique saisit souvent d'avantage les détails essentielles fondant la justesse d'un résultat, de l'autre ses démonstrations sont laborieuse et toute espace non-euclidien nécessite dans ce cadre une nouvelle axiomatisation alors que la conception <<affine>> permet d'envisager d'un coup une ensemble conséquent d'espace non-euclidien. Aujourd'hui c'est la vision <<affine>> qui s'impose dans la recherche en mathématique et en physique.   

%Comme le note Gilbert Asac dans << \href{https://publimath.univ-irem.fr/numerisation/LY/ILY98002/ILY98002.pdf}{L'enseignement de la géométrie au Collège et au Lycée} >>, une contradiction apparaît entre d'une part l'insistance au programme, en Licence et à la préparation du capes et de l'agrégation d'une conception <<affine>> de la géométrie et d'autre part la nécessité d'enseigner une conception <<synthétique>> au collège, puis <<analytique>> (3em-Lycée) mais dérivée de la structure <<synthétique>> à la Euclide. L'enseignement de la géométrie au collège et au lycée est de forte importance car c'est à travers elle que se transmet la logique et le raisonnement. De plus un Enseignement à la Euclide, sans déconsidérer l'effort considérable et la porté des résultat obtenue dés cette époque, ne peut plus satisfaire un professeur de mathématique moderne qui y verra partout les manque et les raccourcis \footnote{C'est encore pire dans les ouvrages données aux professeurs et aux élèves...}. Si ces raccourcis peuvent trompé la plupart des élèves, dont les jurons dans l'effort conceptuelles pousse souvent à se servir d'un résultat plutôt qu'a le comprendre, il ne peuvent qu'entraver partout dans sa préparation le professeur soucieux de la transition de l'effort logico-déductif total. Si on ne se résout pas à l'introduction des structures affines bien plus tôt dans la scolarité comme avais put le tenter les programme de math moderne. Alors une dérivation de la géométrie d'Euclide à la Hilbert semble résoudre cette contradiction. D'un coté le système est rigoureux et de l'autre il ne nécessite pas l'emploie de la structure affine et de l'artillerie d'Algébre linéaire qui s'y rattache. De plus une dérivation en premier lieu des résultats n'utilisant pas l'axiome des parallèles permet de discuter de distinguer le moment ou la géométrie Euclidienne introduit sa particularité et ce qu'elle a en partage avec toute les autres. 